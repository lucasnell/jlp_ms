\documentclass[12pt,]{article}
\usepackage{lmodern}
\usepackage{setspace}
\setstretch{1.5}
\usepackage{amssymb,amsmath}
\usepackage{ifxetex,ifluatex}
\usepackage{fixltx2e} % provides \textsubscript
\ifnum 0\ifxetex 1\fi\ifluatex 1\fi=0 % if pdftex
  \usepackage[T1]{fontenc}
  \usepackage[utf8]{inputenc}
\else % if luatex or xelatex
  \ifxetex
    \usepackage{mathspec}
  \else
    \usepackage{fontspec}
  \fi
  \defaultfontfeatures{Ligatures=TeX,Scale=MatchLowercase}
\fi
% use upquote if available, for straight quotes in verbatim environments
\IfFileExists{upquote.sty}{\usepackage{upquote}}{}
% use microtype if available
\IfFileExists{microtype.sty}{%
\usepackage{microtype}
\UseMicrotypeSet[protrusion]{basicmath} % disable protrusion for tt fonts
}{}
\usepackage[margin=1in,letterpaper]{geometry}
\usepackage{hyperref}
\PassOptionsToPackage{usenames,dvipsnames}{color} % color is loaded by hyperref
\hypersetup{unicode=true,
            pdftitle={jackelope: A swift, versatile molecular evolution and sequencing simulator},
            colorlinks=true,
            linkcolor=Maroon,
            citecolor=Blue,
            urlcolor=Blue,
            breaklinks=true}
\urlstyle{same}  % don't use monospace font for urls
\usepackage{longtable,booktabs}
\usepackage{graphicx,grffile}
\makeatletter
\def\maxwidth{\ifdim\Gin@nat@width>\linewidth\linewidth\else\Gin@nat@width\fi}
\def\maxheight{\ifdim\Gin@nat@height>\textheight\textheight\else\Gin@nat@height\fi}
\makeatother
% Scale images if necessary, so that they will not overflow the page
% margins by default, and it is still possible to overwrite the defaults
% using explicit options in \includegraphics[width, height, ...]{}
\setkeys{Gin}{width=\maxwidth,height=\maxheight,keepaspectratio}
\IfFileExists{parskip.sty}{%
\usepackage{parskip}
}{% else
\setlength{\parindent}{0pt}
\setlength{\parskip}{6pt plus 2pt minus 1pt}
}
\setlength{\emergencystretch}{3em}  % prevent overfull lines
\providecommand{\tightlist}{%
  \setlength{\itemsep}{0pt}\setlength{\parskip}{0pt}}
\setcounter{secnumdepth}{0}
% Redefines (sub)paragraphs to behave more like sections
\ifx\paragraph\undefined\else
\let\oldparagraph\paragraph
\renewcommand{\paragraph}[1]{\oldparagraph{#1}\mbox{}}
\fi
\ifx\subparagraph\undefined\else
\let\oldsubparagraph\subparagraph
\renewcommand{\subparagraph}[1]{\oldsubparagraph{#1}\mbox{}}
\fi

%%% Use protect on footnotes to avoid problems with footnotes in titles
\let\rmarkdownfootnote\footnote%
\def\footnote{\protect\rmarkdownfootnote}

%%% Change title format to be more compact
\usepackage{titling}

% Create subtitle command for use in maketitle
\providecommand{\subtitle}[1]{
  \posttitle{
    \begin{center}\large#1\end{center}
    }
}

\setlength{\droptitle}{-2em}

  \title{jackelope: A swift, versatile molecular evolution and sequencing
simulator}
    \pretitle{\vspace{\droptitle}\centering\huge}
  \posttitle{\par}
    \author{}
    \preauthor{}\postauthor{}
    \date{}
    \predate{}\postdate{}
  

\begin{document}
\maketitle

\hypertarget{abstract}{%
\subsection{Abstract}\label{abstract}}

High-throughput sequencing (HTS) is central to the study of population
genomics. Choices in sampling design for sequencing projects can include
sequencing platform, depth of coverage, and number of individuals to
sample. These choices are most often informed by previous work on highly
diverged species, which ignores species- and population-specific genomic
characteristics, demographies, and evolutionary histories. Simulating
sequencing based on available genomic data better informs sampling
strategies. However, most current methods provide only rudimentary ways
to simulate population structure and variation in coverage among sites.
Here I present the R package \texttt{jackelope} that efficiently (i)
reads and simulates reference genomes; (ii) generates variants using
summary statistics, phylogenies, Variant Call Format (VCF) files, and
coalescent simulations---the latter of which can include selection,
recombination, and demographic fluctuations; (iii) simulates sequencing
error, mapping qualities, multiplexing, and optical/PCR duplicates; and
(iv) writes outputs to standard file formats. \texttt{jackelope} can
simulate single, paired-end, or mate-pair Illumina reads, as well as
reads from Pacific BioSciences. Most functions are written in C++ to
improve performance, and I employed OpenMP to allow for parallel
processing. \texttt{jackelope} is available on GitHub
(\url{https://github.com/lucasnell/jackelope}).

\textbf{Keywords:} sequencing simulator, population genomics,
high-throughput sequencing, Illumina, Pacific Biosciences, Pool-seq

\hypertarget{introduction}{%
\subsection{Introduction}\label{introduction}}

High-throughput sequencing (HTS) is a cost-effective approach to
generate vast amounts of genomic data and has revolutionized the study
of genomes (Metzker, 2009). Large datasets combined with increased error
rates---compared to Sanger sequencing---make bioinformatic pipelines an
important aspect of research using HTS. Many bioinformatic tools exist,
and new programs that are more accurate and computationally efficient
are constantly being developed. To test these tools against known
parameter values, in silico simulation of genomic data is needed.
Although there are many sequence simulators currently available
(reviewed in Escalona, Rocha, \& Posada, 2016), most have only
rudimentary ways to generate population-level data. Events like
population-size changes, selection, or population structure can
drastically change null expectations for sequence data, but including
these possibilities is impossible with most current methods.

In the present paper I introduce \texttt{jackelope}, the first available
HTS simulator in the R (R Core Team, 2019) environment. Designed for
efficient memory use, flexibility, and speed, \texttt{jackelope}
combines the functionality of an HTS simulator with that of a molecular
phylogenetics simulator. Genomes can be derived from FASTA files or
simulated in silico. \texttt{jackelope} can create variants from the
reference genome based on basic population-genomic summary statistics,
phylogenies, Variant Call Format (VCF) files, or coalescent simulations.
These variants can be simulated based on one of several popular
molecular-evolution models. \texttt{jackelope} simulates single,
paired-ended, or mate-pair reads on the Illumina platform, and it
generates Pacific Biosciences (PacBio) reads. Both types of sequencing
output to FASTQ files that can be optionally compressed.

After outlining the methods, I demonstrate the usefulness of
\texttt{jackelope} in informing study design through three common usage
examples.

\hypertarget{features-and-methods}{%
\subsection{Features and methods}\label{features-and-methods}}

Most code is written in C++ and interfaces with R using the
\texttt{Rcpp} package (Eddelbuettel \& François, 2011). I used OpenMP to
allow for parallel processing and the \texttt{PCG} family of
thread-safe, pseudo-random number generators (O'Neill, 2014). Package
\texttt{RcppProgress} provides the thread-safe progress bar (Forner,
2018). An overview of the methods are show in Figure
@ref(fig:jackelope-overview).

\hypertarget{the-ref_genome-class}{%
\subsubsection{\texorpdfstring{The \texttt{ref\_genome}
class}{The ref\_genome class}}\label{the-ref_genome-class}}

Haploid reference genomes are represented by the class
\texttt{ref\_genome}, an R6 (Chang, 2019) class that acts as a wrapper
around a pointer to an underlying C++ object that stores all the
sequence information. They can be generated from FASTA files using the
function \texttt{read\_fasta}. This function also accepts FASTA index
files---created using \texttt{samtools\ faidx}---for faster processing.
Both FASTA and index files can be either uncompressed or compressed
using \texttt{gzip}. If a reference genome is not available, the
\texttt{create\_genome} function creates a reference genome of given
equilibrium nucleotide distributions, and mean and standard deviation of
the sequence-length distribution. I draw sequence lengths from a gamma
distribution (X. Li et al., 2011).

The access provided by the R class \texttt{ref\_genome} is designed to
both maximize flexibility and minimize copying and the chances of
printing extremely large strings to the console. Methods in
\texttt{ref\_genome} allow the user to view the number of sequences,
sequence sizes, sequence names, and individual sequence strings. Users
can also edit sequence names and remove one or more sequences by name.
Method \texttt{filter\_sequences} filters genomes by the minimum
sequence size or by the smallest sequence that retains a given
proportion of total reference sequence if sequences are sorted by
descending size. Using method \texttt{merge\_sequences}, users can
shuffle reference sequences and merge them into one. Method
\texttt{replace\_Ns} replaces any \texttt{N}s in the reference sequence
with random nucleotides, optionally sampled with weights provided by the
user. Reference genomes can be written to FASTA files using the
\texttt{write\_fasta} function.

\hypertarget{the-mevo-class}{%
\subsubsection{\texorpdfstring{The \texttt{mevo}
class}{The mevo class}}\label{the-mevo-class}}

The \texttt{mevo} class stores molecular evolution information for use
in the \texttt{create\_variants} function, and is created using the
\texttt{make\_mevo} function. Molecular evolution can include
substitutions, indels (insertions and deletions) of lengths up to
\(2^{31} - 1\), and variation in mutation rates among sites.

The following substitution models can be employed: TN93 (Tamura \& Nei,
1993), JC69 (Jukes \& Cantor, 1969), K80 (Kimura, 1980), F81
(Felsenstein, 1981), HKY85 (Hasegawa, Kishino, \& Yano, 1985; Hasegawa,
Yano, \& Kishino, 1984), F84 (Thorne, Kishino, \& Felsenstein, 1992),
GTR (Tavar\a'e, 1986), and UNREST (Z. B. Yang, 1994). If using the
UNREST model, equilibrium nucleotide frequencies (\(\mathbf{\pi}\)) are
calculated by solving for \(\mathbf{\pi} \mathbf{Q} = 0\), where
\(\mathbf{Q}\) is the substitution rate matrix. This is done by finding
the left eigenvector of \(\mathbf{Q}\) that corresponds to the
eigenvalue closest to zero.

Insertions and deletions are generated the same and have the same
requirements for inputs in the \texttt{make\_mevo} function. They first
require an overall rate parameter, which is for the sum among all
nucleotides; indel rates do not differ among nucleotides. Function
\texttt{make\_mevo} also requires information about the relative rates
of indels of different sizes, which can be provided in 3 different ways.
First, rates can be proportional to \(\exp(-u)\) for indel length \(u\)
from 1 to the maximum possible length, \(M\) (Albers et al., 2010).
Second, rates can be generated from a Lavalette distribution, where the
rate for length \(u\) is proportional to
\(\left[{u M / (M - u + 1)}\right]^{-a}\) (Fletcher \& Yang, 2009).
Third, relative rates can be specified directly by providing a
length-\(M\) numeric vector of positive values.

Among-site variation in mutation rates is included either by generating
gamma distances (\(\gamma\)) from a distribution or by passing them
manually. The overall mutation rate is \(\gamma ~ q_{0}\), where
\(q_{0}\) is the base mutation rate determined only by the nucleotide.
Gamma distances are generated from a Gamma distribution with a fixed
mean of 1 and with a shape parameter provided by the user. Users can
also pass a list of matrices, one for each reference sequence, with a
gamma distance and end point for each sequence region. The gamma
distances can optionally be written to a BED file.

\hypertarget{the-variants-class}{%
\subsubsection{\texorpdfstring{The \texttt{variants}
class}{The variants class}}\label{the-variants-class}}

Variants from the reference genome are represented as haploid genomes in
the \texttt{variants} class. Similarly to \texttt{ref\_genome}, this R6
class wraps a pointer to a C++ object that stores all the information,
and it was designed to prevent copying of large objects in memory. The
underlying C++ class also does not store whole variant genomes, but
rather just their mutation information---this dramatically reduces
memory usage. Variants can be created in five different ways, all of
which are encompassed in the \texttt{create\_variants} function.

The first two methods directly specify numbers and locations of
mutations and therefore do not require any phylogenetic methods in
\texttt{jackalope}. First, a variant call format (VCF) file can directly
specify mutations for each variant. This method works using the
\texttt{vcfR} package (Knaus \& Grünwald, 2016, 2017) and is the only
method that does not require a \texttt{mevo} object. Second, matrices of
segregating sites from coalescent output can provide the locations of
mutations. A \texttt{mevo} object then provides the type of mutation at
each site. The segregating-site information can take the form of (1) a
coalescent-simulator object from the \texttt{scrm} (Paul R. Staab, Sha
Zhu, Dirk Metzler, \& Gerton Lunter, 2015) or \texttt{coala} (Paul R.
Staab \& Dirk Metzler, 2016) package, or (2) a file containing output
from a coalescent simulator in the format of the \texttt{ms} (Hudson,
2002) or \texttt{msms} (Ewing \& Hermisson, 2010) programs.

The last three methods require simulations along phylogenetic trees,
which is outlined in the following paragraph. In the first of these
phylogenetic methods, phylogenetic tree(s) can be directly input from
either \texttt{phylo} object(s) or NEWICK file(s). One tree can be used
for all genome sequences, or each sequence can use a separate tree.
Alternatively, users can pass an estimate for \(\theta\) (the
population-scaled mutation rate). A random coalescent tree is first
generated using the \texttt{rcoal} function in the \texttt{ape} package
(Paradis \& Schliep, 2018). Then, its total tree length is scaled to be
\(\theta / \mu \sum_{i=1}^{n-1}{1 / i}\) for \(n\) variants and an
equilibrium mutation rate of \(\mu\). The last method allows for
simulation of recombination by simulating along gene trees that can
differ both within and among reference sequences. As for coalescent
segregating sites, gene trees can be from \texttt{scrm} or
\texttt{coala} objects, or from \texttt{ms}-style output files.

Variants are simulated along tree branches by generating exponential
wait times for the Markov ``jump'' chain for each sequence, where the
rate of the exponential distribution is the sum of mutation rates for
each nucleotide in the sequence (Z. Yang, 2006). At each jump, a
position on the sequence is sampled with a probability proportional to
the mutation rate for that nucleotide. To sample positions, I use
weighted reservoir sampling with exponential jumps (Efraimidis \&
Spirakis, 2006). To optionally improve performance for sampling long
sequences, a number of positions can be first uniformly sampled using
``Algorithm D'' by Vitter (1984), followed by weighted sampling by
rates. After sampling a position, a mutation type is sampled with
probabilities proportional to the rate of each mutation type for the
nucleotide present at the sampled position. I used alias sampling
(Kronmal \& Peterson, 1979; Walker, 1974) for sampling mutation types.
Jumps are performed until the summed length of all jumps is greater than
the branch length.

Methods in \texttt{variants} allow the user to view the number of
sequences/variants, variant sequence sizes, sequence/variant names, and
individual variant-sequence strings. Users can also edit variant names,
remove one or more variants by name, and manually add mutations. Variant
information can be written to VCF files using the \texttt{write\_vcf}
function, where each variant can optionally be considered one of
multiple haplotypes for samples with ploidy levels \textgreater{} 1.

\hypertarget{simulate-sequencing-data}{%
\subsubsection{Simulate sequencing
data}\label{simulate-sequencing-data}}

Both R6 classes \texttt{ref\_genome} and \texttt{variants} can be input
to the sequencing functions \texttt{illumina} and \texttt{pacbio}. If
providing a \texttt{variants} object, you can specify sampling weights
for each variant to simulate the library containing differing amounts of
DNA from each. Both methods also allow for a probability of read
duplication, which might occur due to PCR in either method and from
optical duplicates in Illumina sequencing. Reads can be written using
multiple threads by having a read ``pool'' for each thread and having
pools write to file only when they are full. This reduces conflicts that
occur when multiple threads attempt to write to disk at the same time.
The size of a ``full'' pool can be adjusted, and larger sizes should
increase both speed and memory usage. Reads are output to FASTQ files,
optionally with \texttt{gzip} compression.

Function \texttt{illumina} simulates single, paired-ended, or mate-pair
Illumina reads, while \texttt{pacbio} simulates reads from the Pacific
Biosciences (PacBio) platform. Illumina read simulation is based on the
ART program (Huang, Li, Myers, \& Marth, 2011), and PacBio read
simulation is based on SimLoRD (Stöcker, Köster, \& Rahmann, 2016).
Function inputs emulate the program they were based on.

\hypertarget{performance}{%
\subsection{Performance}\label{performance}}

Performance was tested on a 2013 MacBook Pro running macOS High Sierra
with a 2.6GHz Intel Core i5 processor and 8 GB RAM. I compared the
performance of \texttt{jackalope} functions to those in other packages
based on the time and maximum RAM required for each to perform its task.
RAM usage is presented as the maximum usage during an R session where
the only actions taken were to load the necessary package and run the
function. Thus overhead associated with loading a package is included.
For functions in packages outside \texttt{jackalope}, I use the R
convention of displaying them as the package name, two colons, then the
function name (i.e., \texttt{Package::Function}).

\hypertarget{conclusion}{%
\subsection{Conclusion}\label{conclusion}}

\texttt{jackalope} outperforms current programs while providing a more
flexible platform. This package should inform research design for
projects employing HTS, particularly those in population genomics.
Output from \texttt{jackalope} will help develop more specific
sequencing goals in funding applications and estimate the power of a
given sequencing design. Furthermore, \texttt{jackalope} can be used to
test bioinformatic pipelines under assumptions of much more complex
demographic histories than current HTS simulation platforms allow.

\hypertarget{references}{%
\section*{References}\label{references}}
\addcontentsline{toc}{section}{References}

\hypertarget{refs}{}
\leavevmode\hypertarget{ref-Albers_2010}{}%
Albers, C. A., Lunter, G., MacArthur, D. G., McVean, G., Ouwehand, W.
H., \& Durbin, R. (2010). Dindel: Accurate indel calls from short-read
data. \emph{Genome Research}, \emph{21}(6), 961--973.
\url{https://doi.org/10.1101/gr.112326.110}

\leavevmode\hypertarget{ref-Chang_2019}{}%
Chang, W. (2019). \emph{R6: Encapsulated classes with reference
semantics}. Retrieved from \url{https://CRAN.R-project.org/package=R6}

\leavevmode\hypertarget{ref-Eddelbuettel_2011}{}%
Eddelbuettel, D., \& François, R. (2011). Rcpp: Seamless R and C++
integration. \emph{Journal of Statistical Software}, \emph{40}(8),
1--18. \url{https://doi.org/10.18637/jss.v040.i08}

\leavevmode\hypertarget{ref-Efraimidis_2006}{}%
Efraimidis, P. S., \& Spirakis, P. G. (2006). Weighted random sampling
with a reservoir. \emph{Information Processing Letters}, \emph{97}(5),
181--185. \url{https://doi.org/10.1016/j.ipl.2005.11.003}

\leavevmode\hypertarget{ref-Escalona_2016}{}%
Escalona, M., Rocha, S., \& Posada, D. (2016). A comparison of tools for
the simulation of genomic next-generation sequencing data. \emph{Nature
Reviews Genetics}, \emph{17}(8), 459--469.
\url{https://doi.org/10.1038/nrg.2016.57}

\leavevmode\hypertarget{ref-Ewing_2010}{}%
Ewing, G., \& Hermisson, J. (2010). MSMS: A coalescent simulation
program including recombination, demographic structure and selection at
a single locus. \emph{Bioinformatics}, \emph{26}(16), 2064--2065.
\url{https://doi.org/10.1093/bioinformatics/btq322}

\leavevmode\hypertarget{ref-Felsenstein_1981}{}%
Felsenstein, J. (1981). Evolutionary trees from DNA sequences: A maximum
likelihood approach. \emph{Journal of Molecular Evolution},
\emph{17}(6), 368--376. \url{https://doi.org/10.1007/bf01734359}

\leavevmode\hypertarget{ref-Fletcher_2009}{}%
Fletcher, W., \& Yang, Z. (2009). INDELible: A flexible simulator of
biological sequence evolution. \emph{Molecular Biology and Evolution},
\emph{26}(8), 1879--1888. \url{https://doi.org/10.1093/molbev/msp098}

\leavevmode\hypertarget{ref-Forner_2018}{}%
Forner, K. (2018). \emph{RcppProgress: An interruptible progress bar
with openmp support for c++ in r packages}. Retrieved from
\url{https://CRAN.R-project.org/package=RcppProgress}

\leavevmode\hypertarget{ref-Hasegawa_1985}{}%
Hasegawa, M., Kishino, H., \& Yano, T.-a. (1985). Dating of the
human-ape splitting by a molecular clock of mitochondrial DNA.
\emph{Journal of Molecular Evolution}, \emph{22}(2), 160--174.
\url{https://doi.org/10.1007/bf02101694}

\leavevmode\hypertarget{ref-Hasegawa_1984}{}%
Hasegawa, M., Yano, T.-a., \& Kishino, H. (1984). A new molecular clock
of mitochondrial DNA and the evolution of hominoids. \emph{Proceedings
of the Japan Academy, Series B}, \emph{60}(4), 95--98.
\url{https://doi.org/10.2183/pjab.60.95}

\leavevmode\hypertarget{ref-Huang_2011}{}%
Huang, W., Li, L., Myers, J. R., \& Marth, G. T. (2011). ART: A
next-generation sequencing read simulator. \emph{Bioinformatics},
\emph{28}(4), 593--594.
\url{https://doi.org/10.1093/bioinformatics/btr708}

\leavevmode\hypertarget{ref-Hudson_2002}{}%
Hudson, R. R. (2002). Generating samples under a wright-fisher neutral
model of genetic variation. \emph{Bioinformatics}, \emph{18}(2),
337--338. \url{https://doi.org/10.1093/bioinformatics/18.2.337}

\leavevmode\hypertarget{ref-JC69}{}%
Jukes, T. H., \& Cantor, C. R. (1969). Evolution of protein molecules.
In H. N. Munro (Ed.), \emph{Mammalian protein metabolism} (Vol. 3, pp.
21--131). New York: Academic Press.

\leavevmode\hypertarget{ref-Kimura_1980}{}%
Kimura, M. (1980). A simple method for estimating evolutionary rates of
base substitutions through comparative studies of nucleotide sequences.
\emph{Journal of Molecular Evolution}, \emph{16}(2), 111--120.
\url{https://doi.org/10.1007/bf01731581}

\leavevmode\hypertarget{ref-Knaus_2016}{}%
Knaus, B. J., \& Grünwald, N. J. (2016). VcfR: An r package to
manipulate and visualize VCF format data. \emph{BioRxiv}. Retrieved from
\url{http://dx.doi.org/10.1101/041277}

\leavevmode\hypertarget{ref-Knaus_2017}{}%
Knaus, B. J., \& Grünwald, N. J. (2017). VCFR: A package to manipulate
and visualize variant call format data in R. \emph{Molecular Ecology
Resources}, \emph{17}(1), 44--53. Retrieved from
\url{http://dx.doi.org/10.1111/1755-0998.12549}

\leavevmode\hypertarget{ref-Kronmal_1979}{}%
Kronmal, R. A., \& Peterson, A. V. (1979). On the alias method for
generating random variables from a discrete distribution. \emph{The
American Statistician}, \emph{33}(4), 214--218.
\url{https://doi.org/10.1080/00031305.1979.10482697}

\leavevmode\hypertarget{ref-Li_2011}{}%
Li, X., Zhu, C., Lin, Z., Wu, Y., Zhang, D., Bai, G., \ldots{} Yu, J.
(2011). Chromosome size in diploid eukaryotic species centers on the
average length with a conserved boundary. \emph{Molecular Biology and
Evolution}, \emph{28}(6), 1901--1911.
\url{https://doi.org/10.1093/molbev/msr011}

\leavevmode\hypertarget{ref-Metzker_2009}{}%
Metzker, M. L. (2009). Sequencing technologies the next generation.
\emph{Nature Reviews Genetics}, \emph{11}(1), 31--46.
\url{https://doi.org/10.1038/nrg2626}

\leavevmode\hypertarget{ref-Oneill_2014pcg}{}%
O'Neill, M. E. (2014). \emph{PCG: a family of simple fast
space-efficient statistically good algorithms for random number
generation}. Claremont, CA: Harvey Mudd College.

\leavevmode\hypertarget{ref-Paradis_2018}{}%
Paradis, E., \& Schliep, K. (2018). Ape 5.0: An environment for modern
phylogenetics and evolutionary analyses in R. \emph{Bioinformatics},
\emph{35}, 526--528.

\leavevmode\hypertarget{ref-Paul_R._Staab_2016}{}%
Paul R. Staab, \& Dirk Metzler. (2016). Coala: An R framework for
coalescent simulation. \emph{Bioinformatics}.
\url{https://doi.org/10.1093/bioinformatics/btw098}

\leavevmode\hypertarget{ref-Paul_R._Staab_2015}{}%
Paul R. Staab, Sha Zhu, Dirk Metzler, \& Gerton Lunter. (2015). scrm:
Efficiently simulating long sequences using the approximated coalescent
with recombination. \emph{Bioinformatics}, \emph{31}(10), 1680--1682.
Retrieved from
\url{http://bioinformatics.oxfordjournals.org/content/31/10/1680}

\leavevmode\hypertarget{ref-R_Core_Team_2019}{}%
R Core Team. (2019). \emph{R: a language and environment for statistical
computing}. Vienna, Austria: R Foundation for Statistical Computing.
Retrieved from \url{https://www.R-project.org/}

\leavevmode\hypertarget{ref-St_cker_2016}{}%
Stöcker, B. K., Köster, J., \& Rahmann, S. (2016). SimLoRD: Simulation
of long read data. \emph{Bioinformatics}, \emph{32}(17), 2704--2706.
\url{https://doi.org/10.1093/bioinformatics/btw286}

\leavevmode\hypertarget{ref-TN93}{}%
Tamura, K., \& Nei, M. (1993). Estimation of the number of nucleotide
substitutions in the control region of mitochondrial dna in humans and
chimpanzees. \emph{Molecular Biology and Evolution}, \emph{10}(3),
512--526.

\leavevmode\hypertarget{ref-Tavare_1986gtr}{}%
Tavar\a'e, S. (1986). Some probabilistic and statistical problems in the
analysis of DNA sequences. \emph{Lectures on Mathematics in the Life
Sciences}, \emph{17}(2), 57--86.

\leavevmode\hypertarget{ref-Thorne_1992}{}%
Thorne, J. L., Kishino, H., \& Felsenstein, J. (1992). Inching toward
reality: an improved likelihood model of sequence evolution.
\emph{Journal of Molecular Evolution}, \emph{34}(1), 3--16.

\leavevmode\hypertarget{ref-Vitter_1984}{}%
Vitter, J. S. (1984). Faster methods for random sampling.
\emph{Communications of the ACM}, \emph{27}(7), 703--718.
\url{https://doi.org/10.1145/358105.893}

\leavevmode\hypertarget{ref-Walker_1974}{}%
Walker, A. (1974). New fast method for generating discrete random
numbers with arbitrary frequency distributions. \emph{Electronics
Letters}, \emph{10}(8), 127. \url{https://doi.org/10.1049/el:19740097}

\leavevmode\hypertarget{ref-Yang_2006}{}%
Yang, Z. (2006). \emph{Computational molecular evolution}. New York, NY,
USA: Oxford University Press.
\url{https://doi.org/10.1093/acprof:oso/9780198567028.001.0001}

\leavevmode\hypertarget{ref-Yang_1994}{}%
Yang, Z. B. (1994). Estimating the pattern of nucleotide substitution.
\emph{Journal of Molecular Evolution}, \emph{39}(1), 105--111.


\end{document}
