\documentclass[12pt,]{article}
\usepackage{lmodern}
\usepackage{setspace}
\setstretch{1.5}
\usepackage{amssymb,amsmath}
\usepackage{ifxetex,ifluatex}
\usepackage{fixltx2e} % provides \textsubscript
\ifnum 0\ifxetex 1\fi\ifluatex 1\fi=0 % if pdftex
  \usepackage[T1]{fontenc}
  \usepackage[utf8]{inputenc}
\else % if luatex or xelatex
  \ifxetex
    \usepackage{mathspec}
  \else
    \usepackage{fontspec}
  \fi
  \defaultfontfeatures{Ligatures=TeX,Scale=MatchLowercase}
\fi
% use upquote if available, for straight quotes in verbatim environments
\IfFileExists{upquote.sty}{\usepackage{upquote}}{}
% use microtype if available
\IfFileExists{microtype.sty}{%
\usepackage{microtype}
\UseMicrotypeSet[protrusion]{basicmath} % disable protrusion for tt fonts
}{}
\usepackage[margin=1in,letterpaper]{geometry}
\usepackage{hyperref}
\PassOptionsToPackage{usenames,dvipsnames}{color} % color is loaded by hyperref
\hypersetup{unicode=true,
            pdftitle={jackalope: a swift, light-weight phylogenomic and high-throughput sequencing simulator},
            colorlinks=true,
            linkcolor=Maroon,
            citecolor=Blue,
            urlcolor=Blue,
            breaklinks=true}
\urlstyle{same}  % don't use monospace font for urls
\usepackage{longtable,booktabs}
\usepackage{graphicx,grffile}
\makeatletter
\def\maxwidth{\ifdim\Gin@nat@width>\linewidth\linewidth\else\Gin@nat@width\fi}
\def\maxheight{\ifdim\Gin@nat@height>\textheight\textheight\else\Gin@nat@height\fi}
\makeatother
% Scale images if necessary, so that they will not overflow the page
% margins by default, and it is still possible to overwrite the defaults
% using explicit options in \includegraphics[width, height, ...]{}
\setkeys{Gin}{width=\maxwidth,height=\maxheight,keepaspectratio}
\IfFileExists{parskip.sty}{%
\usepackage{parskip}
}{% else
\setlength{\parindent}{0pt}
\setlength{\parskip}{6pt plus 2pt minus 1pt}
}
\setlength{\emergencystretch}{3em}  % prevent overfull lines
\providecommand{\tightlist}{%
  \setlength{\itemsep}{0pt}\setlength{\parskip}{0pt}}
\setcounter{secnumdepth}{5}
% Redefines (sub)paragraphs to behave more like sections
\ifx\paragraph\undefined\else
\let\oldparagraph\paragraph
\renewcommand{\paragraph}[1]{\oldparagraph{#1}\mbox{}}
\fi
\ifx\subparagraph\undefined\else
\let\oldsubparagraph\subparagraph
\renewcommand{\subparagraph}[1]{\oldsubparagraph{#1}\mbox{}}
\fi

%%% Use protect on footnotes to avoid problems with footnotes in titles
\let\rmarkdownfootnote\footnote%
\def\footnote{\protect\rmarkdownfootnote}

%%% Change title format to be more compact
\usepackage{titling}

% Create subtitle command for use in maketitle
\providecommand{\subtitle}[1]{
  \posttitle{
    \begin{center}\large#1\end{center}
    }
}

\setlength{\droptitle}{-2em}

  \title{jackalope: a swift, light-weight phylogenomic and high-throughput sequencing simulator}
    \pretitle{\vspace{\droptitle}\centering\huge}
  \posttitle{\par}
    \author{}
    \preauthor{}\postauthor{}
    \date{}
    \predate{}\postdate{}
  

\begin{document}
\maketitle

{
\hypersetup{linkcolor=black}
\setcounter{tocdepth}{2}
\tableofcontents
}
\hypertarget{abstract}{%
\subsection{Abstract}\label{abstract}}

High-throughput sequencing (HTS) is central to the study of population genomics.
Choices in sampling design for sequencing projects can include
sequencing platform, depth of coverage, and number of individuals to sample.
These choices are most often informed by previous work on highly diverged species,
which ignores species- and population-specific genomic characteristics, demographies,
and evolutionary histories.
Simulating sequencing based on available genomic data better informs sampling
strategies.

However, current methods provide only rudimentary ways to simulate population
histories.
More recent pipelines have been developed that can simulate complex evolutionary

Here I present the R package \texttt{jackalope} that efficiently
(i) reads and simulates reference genomes;
(ii) generates variants using summary statistics, phylogenies,
Variant Call Format (VCF) files, and coalescent simulations---the latter of which can
include selection, recombination, and demographic fluctuations;
(iii) simulates sequencing error, mapping qualities, multiplexing, and
optical/PCR duplicates;
and
(iv) writes outputs to standard file formats.
\texttt{jackalope} can simulate single, paired-end, or mate-pair Illumina reads,
as well as reads from Pacific BioSciences.

Most functions are written in C++ to improve performance, and I employed OpenMP to
allow for parallel processing.
\texttt{jackalope} is available on GitHub
(\url{https://github.com/lucasnell/jackalope}).

\textbf{Keywords:} sequencing simulator, population genomics, phylogenomics,
Illumina, Pacific Biosciences, Pool-seq

\hypertarget{introduction}{%
\subsection{Introduction}\label{introduction}}

High-throughput sequencing (HTS) is a cost-effective approach to generate vast amounts
of genomic data and has revolutionized the study of genomes
(Metzker, 2009).
The combination of massive datasets, sequencing errors, and potentially complex
evolutionary histories make bioinformatic pipelines an important aspect of
research using HTS.
Many bioinformatic tools exist, and new programs that are more accurate and
computationally efficient are constantly being developed.
To test these tools against known parameter values, simulation of HTS data is needed.
Although there are many sequence simulators currently available
(reviewed in Escalona, Rocha, \& Posada, 2016),
most have only rudimentary ways to generate variants from a reference genome.

Recently, pipelines have been developed that use multiple programs
to simulate complex evolutionary histories and HTS on the resulting populations
or species.
\texttt{TreeToReads} can simulate sequences along a single phylogenetic tree and output
to Illumina reads (McTavish et al., 2017).
\texttt{NGSphy} can generate sequences from multiple gene trees, and similarly outputs
Illumina reads (Escalona, Rocha, \& Posada, 2018).
These pipelines are quite powerful and useful to those seeking to simulate
HTS data under realistic evolutionary scenarios.
However, this power comes at a cost.
Both pipelines require the installation of multiple programs to use them,
the vast array of features creates a steep learning curve when getting started,
and the lack of integration between inner programs means that the entire process
is not as computationally efficient as a single standalone program.

In the present paper I introduce \texttt{jackalope}, the first available high-throughput
DNA-sequencing simulator for R
(R Core Team, 2019).
Designed for efficient memory use, flexibility, and speed,
\texttt{jackalope} combines the functionality of an HTS simulator with that of a
phylogenomics simulator.
Genomes can be derived from FASTA files or simulated in silico.
\texttt{jackalope} can create variants from the reference genome based on basic
population-genomic summary statistics, phylogenies, gene trees,
Variant Call Format (VCF) files, or matrices of segregating sites.
These variants can be simulated based on any of several popular
molecular-evolution models.
\texttt{jackalope} simulates single, paired-ended, or mate-pair reads on the Illumina platform,
as well as Pacific Biosciences (PacBio) reads.
All information generated by \texttt{jackalope} can be output to standard file formats.

After outlining the methods, I compare the performance of \texttt{jackalope} to that of
other popular programs.
Lastly, I demonstrate the usefulness of \texttt{jackalope} in informing study design
through three common usage examples.

\hypertarget{features-and-methods}{%
\subsection{Features and methods}\label{features-and-methods}}

Most code is written in C++ and interfaces with R using the \texttt{Rcpp} package
(Eddelbuettel \& François, 2011).
I used OpenMP to allow for parallel processing and
the \texttt{PCG} family of thread-safe, pseudo-random number generators
(O'Neill, 2014).
Package \texttt{RcppProgress} provides the thread-safe progress bar
(Forner, 2018).
All input and output files can have \texttt{gzip} or \texttt{bgzip} compression, using the
\texttt{zlib} and
\texttt{htslib} (Li et al., 2009) libraries.
Access to these libraries uses the R packages
\texttt{zlibbioc} (Morgan, 2018) and
\texttt{Rhtslib} (Hayden \& Morgan, 2019)
to improve portability.
An overview of the methods are show in Figure \ref{fig:jackalope-overview}.

\begin{figure}
\includegraphics[width=1\linewidth,height=1\textheight]{figs/jackalope}
\caption{Overview of primary \texttt{jackalope} functions, classes, inputs, and outputs.
  Circles \texttt{sub\_models} and \texttt{vars\_functions} refer to multiple
  functions; see text for details.
  \(\theta\) indicates the population-scaled mutation rate.}
\label{fig:jackalope-overview}
\end{figure}

\hypertarget{the-ref_genome-class}{%
\subsubsection{\texorpdfstring{The \texttt{ref\_genome} class}{The ref\_genome class}}\label{the-ref_genome-class}}

Haploid reference genomes are represented by the class \texttt{ref\_genome}, an
R6 (Chang, 2019) class that acts as a wrapper around a pointer to
an underlying C++ object that stores all the sequence information.
They can be generated from FASTA files using the function \texttt{read\_fasta}.
This function also accepts FASTA index files---created using
\texttt{samtools\ faidx}---for faster processing.
If a reference genome is not available, the \texttt{create\_genome} function creates
a reference genome of given equilibrium nucleotide distributions.
Sequence lengths are drawn from a gamma distribution (Li et al., 2011)
with a mean and standard deviation provided by the user.

The access provided by the R class \texttt{ref\_genome} is designed to both
maximize flexibility and minimize copying and the chances of
printing extremely large strings to the console.
Methods in \texttt{ref\_genome} allow the user to view the
number of sequences, sequence sizes, sequence names, individual sequence strings,
and nucleotide proportions (GC or otherwise).
Users can also edit sequence names and remove one or more sequences by name.
Method \texttt{filter\_sequences} filters genomes by the minimum sequence size or
by the smallest sequence that retains a given proportion of total reference
sequence if sequences are sorted by descending size.
Using method \texttt{merge\_sequences}, users can shuffle reference sequences and
merge them into one.
Method \texttt{replace\_Ns} replaces any \texttt{N}s in the reference sequence with nucleotides
that are sampled with weights proportional to their equilibrium distributions
(provided by the user).
Reference genomes can be written to FASTA files using the \texttt{write\_fasta} function.

\hypertarget{creating-variants}{%
\subsubsection{Creating variants}\label{creating-variants}}

Haploid variants from the reference genome are generated using the
\texttt{create\_variants} function.
To organize a potentially large amount of information that can be input to this function,
I added helper functions that handle parts of the input.
There are helper functions for the high-level method information and
for various aspects of the molecular evolution (substitutions, indels, and among-site
variability in mutation rates).
The first subsection below outlines the higher-level information functions, and
the second subsection details the molecular evolution functions.

\hypertarget{higher-level-method-information}{%
\paragraph{Higher-level method information}\label{higher-level-method-information}}

There are five ways to generate variants from the reference genome, and
each has a function associated with it.
The names of these functions follow the form \texttt{vars\_X} for method \texttt{X},
and information on all of the function can be found in the \texttt{vars\_functions} documentation.
The outputs of these functions are meant to be passed to the \texttt{vars\_info} argument
in \texttt{create\_variants}.

The first two methods directly specify numbers and locations of mutations and
therefore do not require any phylogenomic methods in \texttt{jackalope}.
First, a variant call format (VCF) file can directly specify mutations for each
variant (function \texttt{vars\_vcf}).
This method works using the \texttt{vcfR} package (Knaus \& Grünwald, 2016, 2017) and
is the only method that does not require any molecular-evolution information.
Second, matrices of segregating sites from coalescent output can provide the locations
of mutations (function \texttt{vars\_ssites}).
Molecular evolution information passed to \texttt{create\_variants} then informs the sampling
for the type of mutation at each site.
The segregating-site information can take the form of
(1) a coalescent-simulator object from the \texttt{scrm} (Paul R. Staab, Sha Zhu, Dirk Metzler, \& Gerton Lunter, 2015) or
\texttt{coala} (Paul R. Staab \& Dirk Metzler, 2016) package, or
(2) a file containing output from a coalescent simulator in the format of the
\texttt{ms} (Hudson, 2002) or
\texttt{msms} (Ewing \& Hermisson, 2010) programs.

The last three methods simulate sequences along phylogenetic trees.
In the first of these methods,
phylogenetic tree(s) can be directly input from either \texttt{phylo} object(s) or
NEWICK file(s) (function \texttt{vars\_phylo}).
One tree can be used for all genome sequences, or each sequence can use a separate tree.
In the second method, users can pass an estimate for \(\theta\), the population-scaled
mutation rate (function \texttt{vars\_theta}).
A random coalescent tree is first generated using the \texttt{rcoal} function
in the \texttt{ape} package (Paradis \& Schliep, 2018).
Then, its total tree length is scaled to be
\(\theta / \mu \sum_{i=1}^{n-1}{1 / i}\) for \(n\) variants and an equilibrium
mutation rate of \(\mu\).
The last method allows for simulation of recombination by simulating along
gene trees that can differ both within and among reference sequences
(function \texttt{vars\_gtrees}).
As for simulations using coalescent segregating sites, gene trees can be
from \texttt{scrm} or \texttt{coala} objects, or from \texttt{ms}-style output files.

For the phylogenomic methods, variants are simulated along tree branches by
generating exponential wait times
for the Markov ``jump'' chain for each sequence,
where the rate of the exponential distribution is the sum of mutation rates for
each nucleotide in the sequence (Yang, 2006).
At each jump, a position on the sequence is sampled with a probability proportional to
the mutation rate for that nucleotide.
To sample positions, I use weighted reservoir sampling with
exponential jumps (Efraimidis \& Spirakis, 2006).
To optionally improve performance for sampling long sequences,
a number of positions can be first uniformly sampled using
``Algorithm D'' by Vitter (1984), followed by weighted
sampling by rates.
After sampling a position, a mutation type is sampled with probabilities
proportional to the rate of each mutation type for the nucleotide
present at the sampled position.
I used alias sampling
(Kronmal \& Peterson, 1979; Walker, 1974)
for sampling mutation types.
Jumps are performed until the summed length of all jumps is greater than the
branch length.

\hypertarget{molecular-evolution}{%
\paragraph{Molecular evolution}\label{molecular-evolution}}

The user can include substitutions, indels (insertions and deletions),
and variation in mutation rates among sites.
A table of rates from Sung et al. (2016) is included in the package
as a guide to users.
Molecular evolution information is provided to the \texttt{create\_variants}
function through organizing functions that are specific to the type of information.
Substitutions use the \texttt{sub\_models} group of functions,
indels use the \texttt{indels} function,
and among-site variation uses the \texttt{site\_var} function.

Each substitution model uses its own function of the pattern \texttt{sub\_M} for model \texttt{M},
and the provided \texttt{sub\_models} documentation includes information to help
users choose among them.
The following substitution models can be employed:
TN93 (Tamura \& Nei, 1993),
JC69 (Jukes \& Cantor, 1969),
K80 (Kimura, 1980),
F81 (Felsenstein, 1981),
HKY85 (Hasegawa, Kishino, \& Yano, 1985; Hasegawa, Yano, \& Kishino, 1984),
F84 (Thorne, Kishino, \& Felsenstein, 1992),
GTR (Tavar\a'e, 1986),
and UNREST (Yang, 1994).
If using the UNREST model, equilibrium nucleotide frequencies (\(\mathbf{\pi}\)) are
calculated by solving for \(\mathbf{\pi} \mathbf{Q} = 0\), where \(\mathbf{Q}\) is the
substitution rate matrix.
This is done by finding the left eigenvector of \(\mathbf{Q}\) that
corresponds to the eigenvalue closest to zero.

Providing insertion and deletion information is organized in the \texttt{indels} function.
It first requires an overall rate parameter, which is for the sum among all
nucleotides; indel rates do not differ among nucleotides.
The \texttt{indels} function also requires information about the relative rates of
indels of different sizes, which can be provided in 3 different ways.
First, rates can be proportional to \(\exp(-u)\) for indel length \(u\) from
1 to the maximum possible length, \(M\) (Albers et al., 2010).
Second, rates can be generated from a Lavalette distribution,
where the rate for length \(u\) is proportional to
\(\left[{u M / (M - u + 1)}\right]^{-a}\) (Fletcher \& Yang, 2009).
Third, relative rates can be specified directly by providing a length-\(M\)
numeric vector of positive values.
The maximum size of indels is 1 Mb.

Among-site variation in mutation rates, specified in the \texttt{site\_var} function,
is included either by generating gamma distances (\(\gamma\)) from a distribution
or by passing them manually.
The overall mutation rate is \(\gamma ~ q_{0}\), where
\(q_{0}\) is the base mutation rate determined only by the nucleotide.
Gamma distances are generated from a Gamma distribution with a fixed mean of 1
and with a shape parameter provided by the user.
Users can also pass a list of matrices, one for each reference sequence,
with a gamma distance and end point for each sequence region.
The gamma distances can optionally be written to a BED file.

\hypertarget{the-variants-class}{%
\subsubsection{\texorpdfstring{The \texttt{variants} class}{The variants class}}\label{the-variants-class}}

Haploid variant information is stored in the \texttt{variants} class.
Similarly to \texttt{ref\_genome}, this R6 class wraps a pointer to a C++ object
that stores all the information, and it was designed to prevent copying of large
objects in memory.
The underlying C++ class also does not store whole variant genomes, but
rather just their mutation information---this dramatically reduces memory usage.
Methods in \texttt{variants} allow the user to view the number of sequences/variants,
variant sequence sizes, sequence/variant names, and individual variant-sequence strings.
Users can also edit variant names, remove one or more variants by name,
and manually add mutations.
Variant information can be written to VCF files using the \texttt{write\_vcf} function,
where each variant can optionally be considered one of multiple haplotypes for
samples with ploidy levels \textgreater{} 1.
Alternatively, variant genomes can be written to separate FASTA files
using \texttt{write\_fasta}.

\hypertarget{simulate-sequencing-data}{%
\subsubsection{Simulate sequencing data}\label{simulate-sequencing-data}}

Both R6 classes \texttt{ref\_genome} and \texttt{variants} can be input to the sequencing functions
\texttt{illumina} and \texttt{pacbio}.
Sequences from which reads are derived are sampled with weights proportional to
their length.
If a \texttt{variants} object is provided, individual variants are sampled with equal
probabilities by default.
Alternatively, the user can specify sampling weights for each variant
to simulate the library containing differing amounts of DNA from each.
Both methods also allow for a probability of read duplication, which might occur
due to PCR in either method and from optical duplicates in Illumina sequencing.
Reads can be written using multiple threads by having a read ``pool'' for each thread
and having pools write to file only when they are full.
This reduces conflicts that occur when multiple threads attempt to write to disk
at the same time.
The size of a ``full'' pool can be adjusted, and larger sizes should increase both
speed and memory usage.
Reads are output to FASTQ files.

Function \texttt{illumina} simulates single, paired-ended, or mate-pair Illumina reads,
while \texttt{pacbio} simulates reads from the Pacific Biosciences (PacBio) platform.
Illumina read simulation is based on the ART program
(Huang, Li, Myers, \& Marth, 2011), and
PacBio read simulation is based on SimLoRD (Stöcker, Köster, \& Rahmann, 2016).
Each was re-coded in C++ to more seemlessly integrate into \texttt{jackalope}.
Function inputs emulate the program they were based on.

\hypertarget{performance}{%
\subsection{Performance}\label{performance}}

Performance was tested on a 2013 MacBook Pro running macOS Mojave with a
2.6GHz Intel Core i5 processor and 8 GB RAM.
Performance was based on the time and maximum RAM required for a program
to perform its task.

\hypertarget{example-usage}{%
\subsection{Example usage}\label{example-usage}}

\hypertarget{conclusion}{%
\subsection{Conclusion}\label{conclusion}}

\texttt{jackalope} outperforms current programs while providing a more flexible platform.

This package should inform research design for projects employing HTS,
particularly those focusing on population genomics or phylogenomics.
Output from \texttt{jackalope} will help develop more specific sequencing goals
in funding applications and estimate the power of a given sequencing design.
Furthermore, \texttt{jackalope} can be used to test bioinformatic pipelines under assumptions of
much more complex demographic histories than most current HTS simulation
platforms allow.

\hypertarget{references}{%
\section*{References}\label{references}}
\addcontentsline{toc}{section}{References}

\hypertarget{refs}{}
\leavevmode\hypertarget{ref-Albers_2010}{}%
Albers, C. A., Lunter, G., MacArthur, D. G., McVean, G., Ouwehand, W. H., \& Durbin, R. (2010). Dindel: Accurate indel calls from short-read data. \emph{Genome Research}, \emph{21}(6), 961--973. \url{https://doi.org/10.1101/gr.112326.110}

\leavevmode\hypertarget{ref-Chang_2019}{}%
Chang, W. (2019). \emph{R6: Encapsulated classes with reference semantics}. Retrieved from \url{https://CRAN.R-project.org/package=R6}

\leavevmode\hypertarget{ref-Eddelbuettel_2011}{}%
Eddelbuettel, D., \& François, R. (2011). Rcpp: Seamless R and C++ integration. \emph{Journal of Statistical Software}, \emph{40}(8), 1--18. \url{https://doi.org/10.18637/jss.v040.i08}

\leavevmode\hypertarget{ref-Efraimidis_2006}{}%
Efraimidis, P. S., \& Spirakis, P. G. (2006). Weighted random sampling with a reservoir. \emph{Information Processing Letters}, \emph{97}(5), 181--185. \url{https://doi.org/10.1016/j.ipl.2005.11.003}

\leavevmode\hypertarget{ref-Escalona_2016}{}%
Escalona, M., Rocha, S., \& Posada, D. (2016). A comparison of tools for the simulation of genomic next-generation sequencing data. \emph{Nature Reviews Genetics}, \emph{17}(8), 459--469. \url{https://doi.org/10.1038/nrg.2016.57}

\leavevmode\hypertarget{ref-Escalona_2018}{}%
Escalona, M., Rocha, S., \& Posada, D. (2018). NGSphy: Phylogenomic simulation of next-generation sequencing data. \emph{Bioinformatics}, \emph{34}(14), 2506--2507. \url{https://doi.org/10.1093/bioinformatics/bty146}

\leavevmode\hypertarget{ref-Ewing_2010}{}%
Ewing, G., \& Hermisson, J. (2010). MSMS: A coalescent simulation program including recombination, demographic structure and selection at a single locus. \emph{Bioinformatics}, \emph{26}(16), 2064--2065. \url{https://doi.org/10.1093/bioinformatics/btq322}

\leavevmode\hypertarget{ref-Felsenstein_1981}{}%
Felsenstein, J. (1981). Evolutionary trees from DNA sequences: A maximum likelihood approach. \emph{Journal of Molecular Evolution}, \emph{17}(6), 368--376. \url{https://doi.org/10.1007/bf01734359}

\leavevmode\hypertarget{ref-Fletcher_2009}{}%
Fletcher, W., \& Yang, Z. (2009). INDELible: A flexible simulator of biological sequence evolution. \emph{Molecular Biology and Evolution}, \emph{26}(8), 1879--1888. \url{https://doi.org/10.1093/molbev/msp098}

\leavevmode\hypertarget{ref-Forner_2018}{}%
Forner, K. (2018). \emph{RcppProgress: An interruptible progress bar with openmp support for c++ in r packages}. Retrieved from \url{https://CRAN.R-project.org/package=RcppProgress}

\leavevmode\hypertarget{ref-Hasegawa_1985}{}%
Hasegawa, M., Kishino, H., \& Yano, T.-a. (1985). Dating of the human-ape splitting by a molecular clock of mitochondrial DNA. \emph{Journal of Molecular Evolution}, \emph{22}(2), 160--174. \url{https://doi.org/10.1007/bf02101694}

\leavevmode\hypertarget{ref-Hasegawa_1984}{}%
Hasegawa, M., Yano, T.-a., \& Kishino, H. (1984). A new molecular clock of mitochondrial DNA and the evolution of hominoids. \emph{Proceedings of the Japan Academy, Series B}, \emph{60}(4), 95--98. \url{https://doi.org/10.2183/pjab.60.95}

\leavevmode\hypertarget{ref-Hayden_2019}{}%
Hayden, N., \& Morgan, M. (2019). \emph{Rhtslib: HTSlib high-throughput sequencing library as an r package}.

\leavevmode\hypertarget{ref-Huang_2011}{}%
Huang, W., Li, L., Myers, J. R., \& Marth, G. T. (2011). ART: A next-generation sequencing read simulator. \emph{Bioinformatics}, \emph{28}(4), 593--594. \url{https://doi.org/10.1093/bioinformatics/btr708}

\leavevmode\hypertarget{ref-Hudson_2002}{}%
Hudson, R. R. (2002). Generating samples under a wright-fisher neutral model of genetic variation. \emph{Bioinformatics}, \emph{18}(2), 337--338. \url{https://doi.org/10.1093/bioinformatics/18.2.337}

\leavevmode\hypertarget{ref-JC69}{}%
Jukes, T. H., \& Cantor, C. R. (1969). Evolution of protein molecules. In H. N. Munro (Ed.), \emph{Mammalian protein metabolism} (Vol. 3, pp. 21--131). New York: Academic Press.

\leavevmode\hypertarget{ref-Kimura_1980}{}%
Kimura, M. (1980). A simple method for estimating evolutionary rates of base substitutions through comparative studies of nucleotide sequences. \emph{Journal of Molecular Evolution}, \emph{16}(2), 111--120. \url{https://doi.org/10.1007/bf01731581}

\leavevmode\hypertarget{ref-Knaus_2016}{}%
Knaus, B. J., \& Grünwald, N. J. (2016). VcfR: An r package to manipulate and visualize VCF format data. \emph{BioRxiv}. Retrieved from \url{http://dx.doi.org/10.1101/041277}

\leavevmode\hypertarget{ref-Knaus_2017}{}%
Knaus, B. J., \& Grünwald, N. J. (2017). VCFR: A package to manipulate and visualize variant call format data in R. \emph{Molecular Ecology Resources}, \emph{17}(1), 44--53. Retrieved from \url{http://dx.doi.org/10.1111/1755-0998.12549}

\leavevmode\hypertarget{ref-Kronmal_1979}{}%
Kronmal, R. A., \& Peterson, A. V. (1979). On the alias method for generating random variables from a discrete distribution. \emph{The American Statistician}, \emph{33}(4), 214--218. \url{https://doi.org/10.1080/00031305.1979.10482697}

\leavevmode\hypertarget{ref-Li_2009}{}%
Li, H., Handsaker, B., Wysoker, A., Fennell, T., Ruan, J., Homer, N., \ldots{} and, R. D. (2009). The sequence alignment/map format and SAMtools. \emph{Bioinformatics}, \emph{25}(16), 2078--2079. \url{https://doi.org/10.1093/bioinformatics/btp352}

\leavevmode\hypertarget{ref-Li_2011}{}%
Li, X., Zhu, C., Lin, Z., Wu, Y., Zhang, D., Bai, G., \ldots{} Yu, J. (2011). Chromosome size in diploid eukaryotic species centers on the average length with a conserved boundary. \emph{Molecular Biology and Evolution}, \emph{28}(6), 1901--1911. \url{https://doi.org/10.1093/molbev/msr011}

\leavevmode\hypertarget{ref-McTavish_2017}{}%
McTavish, E. J., Pettengill, J., Davis, S., Rand, H., Strain, E., Allard, M., \& Timme, R. E. (2017). TreeToReads - a pipeline for simulating raw reads from phylogenies. \emph{BMC Bioinformatics}, \emph{18}(1). \url{https://doi.org/10.1186/s12859-017-1592-1}

\leavevmode\hypertarget{ref-Metzker_2009}{}%
Metzker, M. L. (2009). Sequencing technologies the next generation. \emph{Nature Reviews Genetics}, \emph{11}(1), 31--46. \url{https://doi.org/10.1038/nrg2626}

\leavevmode\hypertarget{ref-Morgan_2018}{}%
Morgan, M. (2018). \emph{Zlibbioc: An r packaged zlib-1.2.5}. Retrieved from \url{http://bioconductor.org/packages/release/bioc/html/Zlibbioc.html}

\leavevmode\hypertarget{ref-Oneill_2014pcg}{}%
O'Neill, M. E. (2014). \emph{PCG: a family of simple fast space-efficient statistically good algorithms for random number generation}. Claremont, CA: Harvey Mudd College.

\leavevmode\hypertarget{ref-Paradis_2018}{}%
Paradis, E., \& Schliep, K. (2018). Ape 5.0: An environment for modern phylogenetics and evolutionary analyses in R. \emph{Bioinformatics}, \emph{35}, 526--528.

\leavevmode\hypertarget{ref-Paul_R._Staab_2016}{}%
Paul R. Staab, \& Dirk Metzler. (2016). Coala: An R framework for coalescent simulation. \emph{Bioinformatics}. \url{https://doi.org/10.1093/bioinformatics/btw098}

\leavevmode\hypertarget{ref-Paul_R._Staab_2015}{}%
Paul R. Staab, Sha Zhu, Dirk Metzler, \& Gerton Lunter. (2015). scrm: Efficiently simulating long sequences using the approximated coalescent with recombination. \emph{Bioinformatics}, \emph{31}(10), 1680--1682. Retrieved from \url{http://bioinformatics.oxfordjournals.org/content/31/10/1680}

\leavevmode\hypertarget{ref-R_Core_Team_2019}{}%
R Core Team. (2019). \emph{R: a language and environment for statistical computing}. Vienna, Austria: R Foundation for Statistical Computing. Retrieved from \url{https://www.R-project.org/}

\leavevmode\hypertarget{ref-St_cker_2016}{}%
Stöcker, B. K., Köster, J., \& Rahmann, S. (2016). SimLoRD: Simulation of long read data. \emph{Bioinformatics}, \emph{32}(17), 2704--2706. \url{https://doi.org/10.1093/bioinformatics/btw286}

\leavevmode\hypertarget{ref-Sung_2016}{}%
Sung, W., Ackerman, M. S., Dillon, M. M., Platt, T. G., Fuqua, C., Cooper, V. S., \& Lynch, M. (2016). Evolution of the insertion-deletion mutation rate across the tree of life. \emph{G3: Genes\(\vert\)Genomes\(\vert\)Genetics}, \emph{6}(8), 2583--2591. \url{https://doi.org/10.1534/g3.116.030890}

\leavevmode\hypertarget{ref-TN93}{}%
Tamura, K., \& Nei, M. (1993). Estimation of the number of nucleotide substitutions in the control region of mitochondrial dna in humans and chimpanzees. \emph{Molecular Biology and Evolution}, \emph{10}(3), 512--526.

\leavevmode\hypertarget{ref-Tavare_1986gtr}{}%
Tavar\a'e, S. (1986). Some probabilistic and statistical problems in the analysis of DNA sequences. \emph{Lectures on Mathematics in the Life Sciences}, \emph{17}(2), 57--86.

\leavevmode\hypertarget{ref-Thorne_1992}{}%
Thorne, J. L., Kishino, H., \& Felsenstein, J. (1992). Inching toward reality: an improved likelihood model of sequence evolution. \emph{Journal of Molecular Evolution}, \emph{34}(1), 3--16.

\leavevmode\hypertarget{ref-Vitter_1984}{}%
Vitter, J. S. (1984). Faster methods for random sampling. \emph{Communications of the ACM}, \emph{27}(7), 703--718. \url{https://doi.org/10.1145/358105.893}

\leavevmode\hypertarget{ref-Walker_1974}{}%
Walker, A. (1974). New fast method for generating discrete random numbers with arbitrary frequency distributions. \emph{Electronics Letters}, \emph{10}(8), 127. \url{https://doi.org/10.1049/el:19740097}

\leavevmode\hypertarget{ref-Yang_2006}{}%
Yang, Z. (2006). \emph{Computational molecular evolution}. New York, NY, USA: Oxford University Press. \url{https://doi.org/10.1093/acprof:oso/9780198567028.001.0001}

\leavevmode\hypertarget{ref-Yang_1994}{}%
Yang, Z. B. (1994). Estimating the pattern of nucleotide substitution. \emph{Journal of Molecular Evolution}, \emph{39}(1), 105--111.


\end{document}
